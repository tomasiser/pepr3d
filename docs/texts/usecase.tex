\chapter{Use case}

\section{Wireframe of the application}

Figure \ref{fig:wireframe} provides a simple wireframe sketch of the application graphical user interface (GUI). The window consists of several usual components:
\begin{itemize}
\item A horizontal toolbar at the top, allowing for a fast selection of tools and file manipulation
\item A 3D preview window, with live preview of the object. This window allows rotating and magnifying the object, as well as the application of selected tools (e.g. painting with a brush). A simple grid and a 3D cross is provided to ensure the user is always aware of object orientation, as it is important for 3D printing.
\item An options window on the right allows the user to customize the settings of the currently selected tool (e.g. selecting the color for a brush).
\end{itemize}

\hspace{-30pt}
\begin{figure}
	\centering
	\includegraphics[scale=0.35]{images/wireframe.png}
	\caption{A simple wireframe sketch of the application.}
	\label{fig:wireframe}
\end{figure}

\section{Workflow}

In this section we describe the intended workflow for a user who has a 3D model he wishes to color and then print on a multicolor FDM printer. This application is intended for users of varying degrees of experience and our goal is to create as easy-to-use application as possible.

\subsection{Importing the model}

First, the user has to import the model he found or created. Clicking on the \textit{import button} creates a dialog, the user selects the 3D model and the model gets loaded. The application should accept at least a few standard formats -- namely Wavefront $.$obj and $.$stl \footnote{https://en.wikipedia.org/wiki/STL\_(file\_format)}, both of which are widespread and well known among the 3D printing community.

The user should also be able to continue on an already existnig project made earlier with Pepr3D.

After the model is loaded, it will be rendered in the 3D preview window. The window allows the user to rotate, zoom in and out and preview the wireframe of the model (rendering only edges and no faces of the model). The user is able to set the number of colors he wants to use (by default $4$ because current Prusa printers support up to four colors). The model is colored with the first color by default.

The user then selects one of the tools from the toolbar. This will bring up the \textit{Tool options} menu on the right hand side allowing the user to customize the tool.

\subsection{Tools}

\subsubsection{Edit history}
%TODO nafukovací feature
The user is always able to revert his last action by using an \textit{Undo button} or a keyboard shortcut. Depending on the technical difficulties, this feature could also persist through different sessions.

\subsubsection{Save as Pepr3D project}
Saves the current project as a Pepr3D project file. Upon re-opening Pepr3D, the user can load the project back and continue the work as if he never left. Does not include exporting the file into a slicer-compatible format.

\subsubsection{Export}
Export the file into a slicer-compatible format. This file is then handed to the slicing program (e.g. Slic3r Prusa Edition we mentioned earlier) and can be printed directly.

\subsubsection{Triangle painter}
After selecting a color, the user can assign said color to triangles he clicks on. Backside filtering is always on, so the user can only ever color a triangle that faces towards him, which should prevent a lot of accidents a lesser experienced user might make.

\subsubsection{Bucket painting}
The user selects a color and by clicking anywhere on the model paints all triangles with selected color until an edge criterion is met. The simplest and most intuitive edge criterion is continuity (a hole stops the bucket spread). Several more criterions could be useful when in 3D, namely the sharpness of the normal (if two neighbouring triangles are at an angle greater than $X$, stop.) or a big gradient in a \textit{shape diameter function} (SDF).

\subsubsection{Automatic segmentation}
Pepr3D fully automatically colors the whole model using the selected colors, according to a edge criterion as discussed in the \textit{Bucket painting} section. The user can then decide if he wants to merge some segments together, reducing the number of colors.

\subsubsection{Semi-automatic segmentation}
The user roughly paints over triangles in areas that should have distinct colors, as indicated by Figure \ref{fig:rabbit}. The program then finishes the coloring by executing a clever flood-fill algorithm utilizing SDF, sharp edges, etc. 

\begin{figure}
	\centering
	\includegraphics[scale=0.75]{images/rabbit.png}
	\caption{Semi-automatic segmentation as seen from the user's perspective. The ears of the rabbit are yellow as indicated by one stroke on each ear. The body is orange as indicated by the stroke on its back. The rabbit's feet are pink -- four pink strokes.}
	\label{fig:rabbit}
\end{figure}

\subsubsection{Brush}
A simple to use brush tool that allows to paint onto the model with a selected color. This tool allows the user to paint finer details, even though the geometry does not include them. For example painting the nose of the rabbit from Figure \ref{fig:rabbit} black -- there is no distinct edges on the nose, but the user can color only the nose by fine strokes of the brush.

The implementation of this tools is harder, because the program needs to adaptively subsample the triangle mesh to allow for finer details. This poses a lot of problems, which will later be discussed in the implementation parts of the document.

\subsubsection{Text}
Using the \textit{tool options} window, the user selects a font and types a custom text into a window. The text gets projected onto the model using some sort of projection transformation (customizable by the user from a limited range of projections). The software also allows extruding the projected text in the direction of the surface normal to create a $3$D effect.

\subsubsection{Triangle subdivision/decimation}
The user selects a section of triangles and then presses either subdivide or decimate, which will either make the geometry more complicated (and smooth), or simpler and more rough. See Figure \ref{fig:decimated} for visual aid.

\begin{figure}
	\centering
	\includegraphics[scale=0.25]{images/decimated_bunny.png}
	\caption{Three stages of triangle numbers. The bunny on the left has the most triangles and most complicated geometry. Several decimations can reduce the number of triangles but also the number of details as shown on the second and third bunny.}
	\label{fig:decimated}
\end{figure}
