\chapter{Execution of the project}

\section{Time schedule}

These are rough estimates of difficulty. We count with approx. 8 hours a week of work for every team member (1 MD = 8 hour of work). The final product should be ready within 7-9 months from the project start.


\subsection{Research}

It has been done during winter and then during spring 2018 with all team members.

\begin{itemize}
\item Comunication with Prusa Research s.r.o., visiting the company, consultation.
\item Comunication and consultation with the supervisor.
\item Familiarization with 3D printing technologies, limitations, existing and similar software, printing first multi mterail models etc. 
\item Preparation of the project assignment.
\end{itemize}


\subsection{Detailed specification}

During summer, approximately 8 MD for every team member.

\begin{itemize}
\item Detailed survey of technologies, continuous consultation with Prusa Research and with the supervisor.
\item Consideration of specific implementation of all tools, studying the necessary topics of computational geometry.
\item Writing detailed binding specification: UI mockups, data structures design, connection of particular modules, design of module inteface, etc.; text correction, printing and submitting the specifiation.
\item Preparation of the repository and division of tasks among team members, icluding time estimation.
\end{itemize}


\subsection{Basic application}

About a month and a half, 6 MD for each team member.

\begin{itemize}
\item \textbf{Basic application with UI:} OpenGL graphical pipeline (3D model view, basic shaders), connection with user inteface -- must be possible to add buttons, texts and text fields, mouse control must work (rotation, zoom).
\item \textbf{Basic module for models:} Import of at least one 3D object file format (.obj, .stl), data structures to represent a multi-color model, support working history (undo/redo), custom file format for loading/storing model -- (de)serialization, export of multi-color model compatible with 3D print slicer (.stl).
\item \textbf{Basic computational geometry:} Geometric data structures (e.g. BVH tree), mouse position detection on model, triangle selection tool, usage of computational libraries (OpenMesh, CGAL, ...).
\item Synchronization between members: plugging modules into UI, preparing UI for other tools, printing bugs and exceptions in UI.
\end{itemize}


\subsection{Complete UI and painting tools}

Approximately three months, 12 MD for every team member.

\begin{itemize}
\item \textbf{Production UI:} Toolbar with icons, taskbar with tool settings, keyboard shortcut support, display current colors on model, option to change printer colors, system windows to load/save files, menu to save file before closing window, opening window before loading model, resize window correctly (scrollbars for content that does not fit or resize fonts, etc.), fullscreen mode, etc.
\item \textbf{Preparing to locate the UI}.
\item \textbf{Multi-color .stl export:} In order that the model could actually be printed is necessary to sufficiently divide it to multiple .stl files by color. It will be necessary to prepare "ddep cut" of the model at least for some elementary shapes, so the slicer can handle it for FDM printing (needs to be experimentally verified).  This task can be arbitrary large, it is almost impossible to produce such a robust export to work for any model -- the greater the robustness, the better.
\item \textbf{Basic tool for triangle painting:} Including back face filtering, UI vizualization, adjustable brush size, etc.
\item \textbf{Flood-fill tools:} Paint bucket tool, automatic and semi-automatic segmentation. It will be needed to implement practical UI for segmentation, so the user could choose the specific colors applied to the model.
\item \textbf{Adaptive triangulation:} See the brush tool section.  This task can be also arbitrarily hard and can be extended because sufficiently robust triangulation is a considerably non-trivial task.
\item \textbf{Brush tool based on triangulation:} See the brush tool section. Also can be arbitrarily hard and can be extended.
\item \textbf{Flat text tool:} See text tool section. The task can also be extended, for example, by different projection of the text (plane, cylinder, circle, etc.) that can by useful in different shapes of models.  Sufficient robustness for different fonts may also be non-trivial (such as fine Japanese characters, UTF-32 symbols, etc. -- may require very fine triangulation, which may cause modification of the implementation).
\item \textbf{Support for 3D text:} Add the option that 2D designed text (see previous tool) can be transferred to 3D. Optionally, add collision detection (so that text does not intersect the model).
\item \textbf{Triangle subdivion/decimation:} See triangle subdivion/decimation section. It can be also exteded, The task has no trivial robust solution.
\end{itemize}



\subsection{Finalization}

Approximately 6 MD for each member. Ideally with project defense. We will try to finish the project as soon as possible.

\begin{itemize}
\item Final version of UI: Ensure that the UI actually matches the mockups, that tools have unified UI, error messages should be meaningful, etc. 
\item Try to compile a project for other platforms (Mac OS, Linux), verify funcionality; eventually, write documentation of/repair platform-dependent bugs (primarily the application have to work on Windows 8/10, but it is appropriate at least make documentation of what should be corrected for functionality on other platforms).
\item Some more time for missing features, bug fixes, extensions
\item Preparation of the final development documentation (ideally, it should copy this specification).
\item Final project testing, preparation of nice examples for demonstration.
\item Preparation of installation packages and user documentation.
\item Project submission and defense.
\end{itemize}


\chapter{Minimal implementation}

This chapter contains minimal project implementation and several possible extensions of tools and features. Not all extended features will be implemented, it very depends on how simple the minimal version will be. We also explicitly mention features that will not be supported in our application.


\section{Implementation extension}

Some project features and tools can be added or extended. This section if focused on minimal and extended states of these features.


\subsection{Menu}

Minimal implementation of menu is a horizontal toolbar at the top of the application. In addition we were thinking of radial menu around cursor, which would show after specific mouse button click. The radial menu has the advantage that the user does not have to move the cursor off the model.


\subsection{Text projection}

In addition to simple planar projection of a text onto a model we considered some more complex projection. For example, cylindrical or spherical projection.


\subsection{3D text collision}

Since we allow the extrusion of the text above the surface itself, the collision between text and another part of the model may occur. To prevent unwanted text extrusion we could add collision detection feature.


\subsection{Model exporting}

Complexity of model exporting can be very different. Minimal implementation will be somehow usable in all situations. We could try to extend it to a possible optimal solution.


\subsection{Subdivision and decimation}

Minimal implementation will include simple triangle subdivision/decimation. As extension we can implement any more sophisticated algorithm for that.

\subsection{Adaptive triangulation}

As mentioned in tool section (Brush) this feature can be arbitrarily difficult. The minimal implementation will contain usable version of adaptive triangulation. Extension may include any better adaptive triangulation.


\section{Not supported features}

In order to avoid any misunderstandings in functionality of our application, there are features that will not be supported. Most of these features can be found in various 3D modeling application for 3D printers.

\begin{itemize}
\item 3D object modelling, sculpting, adding vertices/traingles; there are many other programs that can do it.
\item Repairing 3D model (e.g., filling holes, avoiding model intersection); our model splitting should not break the model or make holes in it.
\item Creating any model support for better 3D printing or generating model infill.
\item Solving any printing problems or setting printer settings.
\item Generating G-code for 3D printers.
\end{itemize}










